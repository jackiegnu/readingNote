\documentclass{article}

\usepackage{titlesec}
\usepackage{titling}
\usepackage[margin=1.25in]{geometry}
%\usepackage{fullpage}
\usepackage{anysize}
\usepackage{url}

\marginsize{1in}{.25in}{.25in}{.25in}

\titleformat{\section}[frame]
{\huge}
{}
{0.25em}
{\filcenter\bfseries}

\titleformat{\subsection}
{\bfseries\Large}
{}
{0em}
{\hspace{-.25in}$\bullet$}

\titleformat{\subsubsection}[runin]
{\bfseries}
{}
{0em}
{}[---]

\titlespacing{\subsubsection}
{0em}{0.25em}{1em}

\renewcommand{\maketitle}{
\begin{center}

{\huge\bfseries\theauthor}

\vspace{.25em}
qujianhua2005@126.com --- \url {https://github.com/jackiegnu}

\vspace{.25em}
\thedate

\end{center}
}

\pagestyle{plain}

%%%%%%%%%%%%%%%%%%%%%%%%%%%%%%%%%%%%%%%%%%%%%%%%%%%%%%%%%%%%%%%%%%%%%%%%%%%%%%%%%%%%
\begin{document}

\title {R\'esum\'e}
\author {Jianhua Qu}
\date{\today}

\maketitle

\section{Technical skills}

\subsection{Work flow}
\subsubsection{Editor}
Vim, an awesome editor I have been using for long time.

\subsubsection{VCS}
git, svn

\subsubsection{Cryptography}
openssl, I used its toolkit a lot in my work such as keypair generation,
        encyption/decryption, digest, signing and so forth

\subsubsection{Misc}
screen, tmux, fzf, etc;
Those help me work efficiently

\subsubsection{Iot related stuff}
\begin{itemize}
\item[1] Iot system: samsung things
\item[2] Protocol: Coap, 6LoWPANs:


\end{itemize}

\subsection{Languages}

\subsubsection{Programming}
C, C++, Java, Golang, ARM-Asm

\subsubsection{Scripting}
Python, Bash, CMake, Makefile, SQL

\subsubsection{Markup}
HTML, CSS, Markdown

\subsection{My Special}

\subsubsection {Common skills}
Linux coding; socket programming; C/C++ programming

\subsubsection {TEE skills}\footnote {TEE: Trusted Execution Enviornment}
ARM Trustzone knowledge; Secure OS development experience;

\subsubsection {Android skills}
Android Application/Framework development experience;

\subsection {Something Learned by Myself just for fun}
\subsubsection {Database}
MySQL, Redis, MangoDB

\subsubsection {Cloud}
docker, kubernetes, AWS EC2(with testing account for free), Aliyun(with my own account)

\section{General skills}
\subsection{Languages}

\subsubsection{can speak and listen}
Chinese, English, Korean

\subsubsection{can read and write}
Chinese, English

\section{Job Experience}
\subsection{March 2008\~{}Present, Samsung}

\subsubsection{Jan 2014\~{}Present}
\textbf{Beijing Samsung Communications Technology Research Co.,Ltd}

Back to Beijing China from Korea, I led two projects, and, I am responsible for 2 applications design, architecture and development.

\begin{itemize}
\item{\textbf{passwordless payment feature---}}
I led second project on which I was a mainly developer, too.

Alipay and WeChat are the giant android application in China.
To support Alipay and Wechat using biometrics such as fingerprint to do passwordless payment,
I develop a middleware to support Alipay and WeChat using biometrics technology on Samsung devices.
Also, I use TEE technology to persist user sensitive data and RSA key-pair.

The architecture of this feature consists of Android AIDL interface, Android Service, Linux daemon, CA\footnote{CA:Client Application} library
and TA\footnote{TA: Trusted Application}.

Those features run in 3 chipset vendors: Qualcomm, Samsung Exynos and MediaTek;

There are 2 passwordless payment stardards from Alipay and Tencent respectively as below:

\begin{itemize}
\item[1]{IIFAA}\footnote{IFAA: \url{http://ifaa.org.cn/}}
(International Internet Finance Authentication Alliance)

\item[2]{SOTER}\footnote{SOTER github: \url{https://github.com/Tencent/soter}}
is a biometric standard as well as a platform held by Tencent.
\end{itemize}

\item{\textbf{GlobalRoaming}}
This was the first project I led, and as a main role of design and development.

It commercialized on Samsung devices since 2015, now, it has been transfer to another team;
This application would be used when users go abroad to access the local data service from local telecom vendor.
We saved IMSI profile data from telecom vendor into TEE side, as well, I implemented USIM related specifications of 3GPP to communicate to MS.
During this project, I refered to or implemented ISO7816-4, ETSI TS 131-102/121, ETSI TS 135 208.

Also, I used cryptography alogrithm, such as RSA, AES, to exchange cipher key and encrypt/decrypt data/cipher.

Those features run in 3 chipset vendor: Qualcomm, Samsung Exynos and MediaTek;

\end{itemize}

\subsubsection{Jan 2012\~{}Dec 2013}
\textbf{Samsung electronics Korea Headquarter}

Joined the Global Mobility to move to Headquarter in Suwon Korea;

\begin{itemize}
\item{}During this two years working in Korea, worked with Korean colleague to develop
\textbf{Secure OS}\footnote{Secure OS: works in the secure mode of ARM with trustzone supported} based on Samsung Chipset Exynos.
This was my first time known about TEE(Trusted Execution Envirnment).
And, we conform with GlobalPlatform {\footnote {GlobalPlatform: https://globalplatform.org/}} specification to implement Secure OS.

\end{itemize}

\subsubsection{March 2008\~{}Dec 2012}
\textbf{Beijing Samsung Communications Technology Research Co.,Ltd}

\begin{itemize}
\item{Android Application developer} since 2010
\item{SAP(Samsung Application Platform) Application developer}, this was deprecated feature phone platform
\end{itemize}

\subsection{July 2006\~{}March 2008}
\subsubsection{TechFaith (NASDAQ: CNTF)}
Linux driver developer

\section{Education}

\subsection{M.A(Mechanical Manufacture and Automation Major)}
\subsubsection{September 2003\~{}July 2006}
China Agricultrue Univerity(\url{https://www.cau.edu.cn/})

\subsection{B.A(Mechanical Design Manufacture and Automation Major)}
\subsubsection{September 1999\~{}July 2003}
HeBei Agricultrue Univerity(\url{http://www.hebau.edu.cn/})

\end{document}
