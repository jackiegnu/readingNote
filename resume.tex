\documentclass{article}

\usepackage{titlesec}
\usepackage{titling}
\usepackage[margin=1.25in]{geometry}
%\usepackage{fullpage}
\usepackage{anysize}
\usepackage{url}

% \marginsize{1in}{.25in}{.25in}{.25in}
\marginsize{1in}{1in}{.25in}{.25in}

\titleformat{\section}[frame]
{\huge}
{}
{0.25em}
{\filcenter\bfseries}

\titleformat{\subsection}
{\bfseries\Large}
{}
{0em}
{\hspace{-.25in}$\bullet$}

\titleformat{\subsubsection}[runin]
{\bfseries}
{}
{0em}
{}[---]

\titlespacing{\subsubsection}
{0em}{0.25em}{1em}

\renewcommand{\maketitle}{
\begin{center}

{\huge\bfseries\theauthor}

\vspace{.25em}
qujianhua2005@126.com

% \vspace{.25em}
% \url {http://59.110.160.106:8081/resume.pdf}

\vspace{.25em}
Chinese, Male, born in Dec.1979, Married, 14 years work experience

\vspace{.25em}
Update by \thedate

\end{center}
}

\pagestyle{plain}

%%%%%%%%%%%%%%%%%%%%%%%%%%%%%%%%%%%%%%%%%%%%%%%%%%%%%%%%%%%%%%%%%%%%%%%%%%%%%%%%%%%%
\begin{document}

\title {R\'esum\'e}
\author {Jianhua Qu}
\date{\today}

\maketitle

\section{Technical skills}

\subsection{Work flow}
\subsubsection{Editor}
Vim, an awesome editor I have been using for long time.

\subsubsection{VCS}
git, svn

\subsubsection{Cryptography}
openssl, I used its toolkit a lot in my work such as keypair generation,
        encyption/decryption, digest, signing and so forth

\subsubsection{shell}
zsh, bash - I am using .oh-my-zsh a lot

\subsubsection{Misc}
screen, tmux, fzf, etc;
Those help me work efficiently

\subsection{Programming Languages}

\subsubsection{Programming}
C, C++, Java, Golang, ARM-Asm

\subsubsection{Scripting}
Python, Bash, CMake, Makefile, SQL

\subsection{My Special}

\subsubsection {Common skills}
Linux coding; Socket programming; C/C++ programming; Desgin Pattern

\subsubsection {TEE skills} \footnote{TEE: Trusted Execution Enviornment}
ARM Trustzone knowledge; Secure OS development experience;

\subsubsection {Android skills}
Android Application/Framework development experience;

\subsection {Something Learned by Myself just for fun}
\subsubsection {Database}
MySQL, Redis, MongoDB

\subsubsection {Cloud}
docker, kubernetes, AWS EC2(with testing account for free), Aliyun(with my own account)

\subsubsection{Iot related stuff}
\begin{itemize}
\item[1] Iot system: samsung things
\item[2] Protocol: Coap, 6LoWPANs:
\end{itemize}


\section{Job Experience}
\subsection{March 2008\~{} December 2020, Samsung Electronics}

\subsubsection{January 2014\~{}December 2020}
\textbf{Beijing Samsung Communications Technology Research Co.,Ltd}

Back to Beijing China from Korea, I led two projects, and, I am responsible for 2 applications design, architecture and development.

\begin{itemize}
\item{\textbf{passwordless payment feature---}}

I led the second project on which I was a mainly developer, too.

This feature has been implemented by 2 middlewares to support both Alipay and Wechat Payment separately, since they are not compatible.
Those 2 middlewares are running only on Samsung Android devices.

Alipay and WeChat are the giant android application in China.
To support Alipay and Wechat using biometrics such as fingerprint to do passwordless payment,
I develop those middlewares to support Alipay and WeChat using biometrics technology on Samsung devices.
Also, I use TEE technology to persist user sensitive data and RSA key-pair.

The architecture of this feature consists of Android AIDL interface, Android Service, Android HIDL interface, Linux service daemon, CA\footnote{CA:Client Application} library
and TA\footnote{TA: Trusted Application}.

Those features run in 3 chipset vendors: Qualcomm, Samsung Exynos and MediaTek;

There are 2 passwordless payment stardards from Alipay and Tencent respectively as below:

\begin{itemize}
\item[1]{IIFAA}\footnote{IFAA: \url{http://ifaa.org.cn/}}
(International Internet Finance Authentication Alliance)

\item[2]{SOTER}\footnote{SOTER github: \url{https://github.com/Tencent/soter}}
is a biometric standard as well as a platform held by Tencent.
\end{itemize}

\item{\textbf{GlobalRoaming}}
GlobalRoaming is a complex Android application which includes user interface logic, Service provider, SoftSIM logic, Middleware between Modem and Applicaton Processor, Modem RIL, Modem Program etc.

This is the first project I led, and as a main role of design and development.

This application would be used when users go abroad to access the local data service from local telecom vendor.
IMSI profile data from telecom vendor are stored into TEE side for security reason, and implemented USIM related specifications of 3GPP to complete communicate to BSS.
During this project, I refered to and implemented ISO7816-4, ETSI TS 131-102/121, ETSI TS 135 208.

Also, I used cryptography alogrithm, such as RSA, AES, to exchange cipher key and encrypt/decrypt data/cipher.

Those features run in 3 chipset vendor: Qualcomm, Samsung Exynos and MediaTek;

It has been commercialized on Samsung devices since 2015, it was transfered to another subsidiary of Samsung;


\end{itemize}

\subsubsection{January 2012\~{}January 2014}
\textbf{Samsung electronics Korea Headquarter}

Joined the Global Mobility Program to move to Headquarter for working in Suwon Korea;

\begin{itemize}
\item{}During this two years working in Korea, worked with Korean colleague to develop
\textbf{Secure OS}\footnote{Secure OS: works in the secure mode of ARM with trustzone supported} based on Samsung Chipset Exynos.
This was my first time known about TEE(Trusted Execution Envirnment).
And, we conform with GlobalPlatform {\footnote {GlobalPlatform: https://globalplatform.org/}} specification to implement Secure OS.

I was responsible for Device Driver model development for secure OS. And, I designed and implemented secure LCD evaluation application. For that, I shared the quarter best employee award with my colleague.

\end{itemize}

\subsubsection{March 2008\~{}January 2012}
\textbf{Beijing Samsung Communications Technology Research Co.,Ltd}

\begin{itemize}
\item{Android Application developer} since 2010
I was a developer and maintainer of Android applicaton of MusicPlayer;

\item{SAP(Samsung Application Platform) Application developer}, this was deprecated feature phone platform.
\end{itemize}

\subsection{July 2006\~{}March 2008, TechFaith (NASDAQ: CNTF)}
Linux driver developer
\begin{itemize}
\item{Linux device driver engineer}

\item{Board bring up developer}

\end{itemize}

\section{Languages}

\subsection{can speak and listen}
Chinese, English, Korean

\subsection{can read and write}
Chinese, English


\section{Education}

\subsection{Master Degree(Mechanical Manufacture and Automation Major)}
\subsubsection{September 2003\~{}July 2006}
China Agricultrue Univerity(\url{https://www.cau.edu.cn/})

\subsection{Bachelor Degree(Mechanical Design Manufacture and Automation Major)}
\subsubsection{September 1999\~{}July 2003}
HeBei Agricultrue Univerity(\url{http://www.hebau.edu.cn/})

% \section{Hobbies \& Interests}
% \subsection{Activities}
% \begin{itemize}
% \item{I love playing table tenis and badminton with my famliy and friends}
% \item{I enjoy the leisure time to play video game with my son}
% \end{itemize}

\section{ Self Introduction}
\begin{itemize}
\item{My wife and son are living in Stockholm Sweden now, so, I am going to move to Sweden for family reunion on January 17, 2021}
\item{I am job hunting now, since I just quit my job from Samsung Electronics(\textit{Beijing China}) where I had been working for 12 years.}

\item{My favorite job is to be a programmer, I like to learn new technical skills and to accept new challenges which could offer me new opportunities.}

\item{It is nice to spend some quality time with my family; playing video game is one of our favorite activities. We also love sport activities such as playing table-tennis, hiking, jogging etc.}
\end{itemize}
\end{document}
